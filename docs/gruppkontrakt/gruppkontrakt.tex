\documentclass{mall}

\newcommand{\version}{Version 1.0}
\author{Albin Dahlén, \url{albda746@student.liu.se}\\
  Filip Ingvarsson, \url{filin764@student.liu.se}}
\title{Gruppkontrakt}
\date{2022-11-10}
\rhead{}


\begin{document}
\projectpage

\section{Hur vi arbetar tillsammans}


\begin{itemize}
\item \textbf{Vilka tider arbetar vi, och vilka tider är vi nåbara utöver detta?}

  Vi arbetar vardagar mellan 8 och 17. På kvällar och helger är det okej att inte vara nåbar.

\item \textbf{Hur kommunicerar vi med varandra? Vilka verktyg/kanaler använder vi?}

  Vi kommunicerar i sal eller med Facebook messenger. 

\item \textbf{Hur gör vi för att ge varandra möjlighet att framföra åsikter och tankar om uppgifter och idéer till arbetet?}

  Vi diskuterar fram den bästa lösningen.

\item \textbf{Hur ofta tar vi paus? Ska vi hjälpas åt att påminna varandra om att ta paus?}

  Vi tar paus vid behov vid individuellt arbete. Vid arbete i sal tar vi paus vid behov.

\item \textbf{Arbetar vi tillsammans med uppgifter, eller var för sig?}

  Vi kommer arbeta både tillsammans och enskilt. Vi arbetar för det mesta enskilt men träffas i sal för att diskutera och planera arbetet.

\item \textbf{Hur bestämmer vi vem som gör vad?}

  Vi har korta mötet vid behov och bestämmer där vad vi ska göra.

\item \textbf{Hur specifierar vi vad som ingår i varje uppgift, och när den är klar?}

  Vi diskuterar under mötena.

\item \textbf{Hur snabbt förväntar vi oss att en uppgift kan vara klar?}

  Bestäms under mötena beroende på uppgiftens omfattning.

\item \textbf{Hur håller vi reda på att uppgifter vi identifierat inte glöms bort?}

  Vi använder gitlabs inbyggda kalender för att hålla reda på de olika delmomentent och milstolpar i
  kodningen. Vi använder sedan ett textdokument för att hålla reda på andra deadlines för projektet.

\end{itemize}

\end{document}
