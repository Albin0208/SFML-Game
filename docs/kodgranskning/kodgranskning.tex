\documentclass{TDP005mall}

\usepackage{graphicx}
\usepackage{float}

\newcommand{\version}{Version 1.0}
\author{Albin Dahlén, \url{albda746@student.liu.se}\\
  Filip Ingvarsson, \url{filin764@student.liu.se}}
\title{Kodgranskningsprotokoll}
\date{2022-12-08}
\rhead{Albin Dahlén\\
Filip Ingvarsson}


\begin{document}
\projectpage
\section{Revisionshistorik}
\begin{table}[!h]
\begin{tabularx}{\linewidth}{|l|X|l|}
\hline
Ver. & Revisionsbeskrivning & Datum \\\hline
1.0 & Första utkast & 22-12-08 \\\hline
\end{tabularx}
\end{table}

\section{Möte}
Före mötet såg vi till att beställargruppen hade tillgång till vår kod på Gitlab samt att vi hade tillgång till deras.
Före själva mötet hade båda grupperna granskat varandras kod och antecknat synpunkter som sedan tog upp på mötet.
Mötet ägde rum 8 December klockan 14:00.

\section{Granskat Projekt}
Spelet som granskades var ett plattformsspel där spelaren styrde en karaktär med ett visst antal liv och målet var att nå slutet av banan.
Under speletsgång möter spelaren fiender som kan skada spelaren om de kolliderar.

Projektet vi granskat, vilka som gjort det kanske vi ska skriva?

\subsection{Kommentering}
Koden saknade nästan helt kommentarer vilket gjorde det väldigt svårt att förstå vad kodstyckena gjorde.


\subsection{Const ref}
Väldigt lite använding av const och referenser

\subsection{Kodupprepning}
Player::left()
Player::right()

\subsection{Variabelnamn}
On\_ground kan skapa förvirring om man inte är insatt i projektet, bättre variabel namn eller bättre kommentering hade löst detta


\section{Vårat Projekt}
\subsection{Kommentering}
Beställarengruppen påpekade att majoriteten av projektet var bra kommenterat, men sedan att visa klasser och delar av kodstycken saknade förklarande kommentarer.
Vi har planerat att när ett tillfredställande resultat uppnåtts ska koden gås igenom och kommentarer ska uppdateras och förbättras eventuellt läggas till om något stycke inte är helt klart.

\subsection{Snake och camel-case}
Blandat olika kodstilar vilket kan göra koden svårare att läsa

\subsection{Projektiler}
Våra projektiler kan för tillfället bara träffa enemies, hur ska då enemies skjuta på spelaren. Vi hade inte kommit tillräckligt långt i vårt spel och har itne implementerat detta än.

\subsection{Testa Programmet}
Inte nån read-me om hur man ska köra programmet, vi har jobbat via clion och de via vscode/terminalen.

\end{document}
