\documentclass{TDP005mall}

\usepackage{graphicx}
\usepackage{float}

\newcommand{\version}{Version 1.0}
\author{Albin Dahlén, \url{albda746@student.liu.se}\\
  Filip Ingvarsson, \url{filin764@student.liu.se}}
\title{Kodgranskningsprotokoll}
\date{2022-12-08}
\rhead{Albin Dahlén\\
Filip Ingvarsson}


\begin{document}
\projectpage
\section{Revisionshistorik}
\begin{table}[!h]
\begin{tabularx}{\linewidth}{|l|X|l|}
\hline
Ver. & Revisionsbeskrivning & Datum \\\hline
1.0 & Första utkast & 22-12-08 \\\hline
\end{tabularx}
\end{table}

\section{Möte}
Före mötet såg vi till att beställargruppen hade tillgång till vår kod på Gitlab samt att vi hade tillgång till deras.
Före själva mötet hade båda grupperna granskat varandras kod och antecknat synpunkter som sedan tog upp på mötet.
Mötet ägde rum 8 December klockan 14:00.

\section{Granskat Projekt}
Spelet som granskades var ett plattformsspel där spelaren styrde en karaktär med ett visst antal liv och målet var att nå slutet av banan.
Under speletsgång möter spelaren fiender som kan skada spelaren om de kolliderar.

Projektet vi granskat, vilka som gjort det kanske vi ska skriva?

\subsection{Kommentering}
Koden saknade nästan helt kommentarer vilket gjorde det väldigt svårt att förstå vad kodstyckena gjorde.


\subsection{Const ref}
Koden saknade const och ref i sina funktioner, detta kan göra att koden gör saker man inte tänkt och kan vara väldigt svårt och tidskrävande att lägga till i slutet av projektet. Programmet tar även mer minne då den kommer skapa kopior till alla istället för att skicka referenser.

\subsection{Kodupprepning}
De hade några funktioner i sina klasser som var väldigt snarlika t.ex, i Move funktioner för player kallar de på Left() och Right() beroende på om användaren klickar på höger eller vänster piltangent. Left och right funktionerna gör dock i princip samma sak, ändrar på samma värden men till olika saker. Ett förslag hade varit att ha det i samma funktion men ta någon parameter som avgör vad som ändras enligt Don't repeat yourself principen.

\subsection{Variabelnamn}
Player har en variabel som heter 'On\_ground' samt en variabel som heter 'is\_jumping'. När vi granskade deras kod innan mötet tänkte vi bara att dessa variabler var motsattsen mot varandra men visade sig att 'on\_ground' hade att göra med vilket håll gravitationen var riktad. Detta missförstånd hade kunnat undvikas vid bättre variabelnamn kanske 'gravity\_down' istället för 'on\_ground', eller kommentering som göra att en ny läsare av koden kan förstå vad som menas.




\section{Vårat Projekt}
\subsection{Kommentering}
Beställarengruppen påpekade att majoriteten av projektet var bra kommenterat, men sedan att visa klasser och delar av kodstycken saknade förklarande kommentarer.
Vi har planerat att när ett tillfredsställande resultat uppnåtts ska koden gås igenom och kommentarer ska uppdateras och förbättras eventuellt läggas till om något stycke inte är helt klart.

\subsection{Formatering}
Beställargruppen påpekade att vi använt oss av flera olika kodstilar,
däribland snake\_case och camelCase vilket gjorde koden jobbigare att läsa då den inte var konsekvent.
Detta är något vi efter påpekan ska gå igenom och fixa så att det är samma standard över hela projektet.

\subsection{Projektiler}
Våra projektiler kan för tillfället bara träffa enemies, hur ska då enemies skjuta på spelaren. Vi hade inte kommit tillräckligt långt i vårt spel och har itne implementerat detta än.

\subsection{Testa Programmet}
Beställargruppen påpekade att det inte fanns någon instruktion hur man körde programmet.
En temporär lösning skapades för att kunna köra programmet.
En faktisk lösning ska lösas snarast så att programmet inte är beroende av Clion för att köras.

\end{document}
