\documentclass{TDP005mall}

\usepackage{graphicx}
\usepackage{float}

\newcommand{\version}{Version 1.0}
\author{Albin Dahlén, \url{albda746@student.liu.se}\\
  Filip Ingvarsson, \url{filin764@student.liu.se}}
\title{Kodgranskningsprotokoll}
\date{2022-12-08}
\rhead{Albin Dahlén\\
Filip Ingvarsson}


\begin{document}
\projectpage
\section{Revisionshistorik}
\begin{table}[!h]
\begin{tabularx}{\linewidth}{|l|X|l|}
\hline
Ver. & Revisionsbeskrivning & Datum \\\hline
1.0 & Första utkast & 22-12-08 \\\hline
\end{tabularx}
\end{table}

\section{Möte}
Tid, plats förberedelse etc

\section{Granskat Projekt}
Projektet vi granskat, vilka som gjort det kanske vi ska skriva?
\subsection{Kommentering}
Kommentering av koden saknades på många ställen

\subsection{Const ref}
Väldigt lite använding av const och referenser

\subsection{Kodupprepning}
De hade några funktioner i sina klasser som var väldigt snarlika t.ex, i Move funktioner för player kallar de på Left() och Right() beroende på om användaren klickar på höger eller vänster piltangent. Left och right funktionerna gör dock i princip samma sak, ändrar på samma värden men till olika saker. Ett förslag hade varit att ha det i samma funktion men ta någon parameter som avgör vad som ändras enligt Don't repeat yourself principen.

\subsection{Variabelnamn}
Player har en variabel som heter 'On\_ground' samt en variabel som heter 'is\_jumping'. När vi granskade deras kod innan mötet tänkte vi bara att dessa variabler var motsattsen mot varandra men visade sig att 'on\_ground' hade att göra med vilket håll gravitationen var riktad. Detta missförstånd hade kunnat undvikas vid bättre variabelnamn kanske 'gravity\_down' istället för 'on\_ground', eller kommentering som göra att en ny läsare av koden kan förstå vad som menas.

\section{Vårat Projekt}
\subsection{Kommentering}
Dåligt kommenterat på vissa ställen, andras ställen hade dock bra kommentering
\subsection{Snake och camel-case}
Blandat olika kodstilar vilket kan göra koden svårare att läsa

\subsection{Projektiler}
Våra projektiler kan för tillfället bara träffa enemies, hur ska då enemies skjuta på spelaren. Vi hade inte kommit tillräckligt långt i vårt spel och har itne implementerat detta än.

\subsection{Testa Programmet}
Inte nån read-me om hur man ska köra programmet, vi har jobbat via clion och de via vscode/terminalen.

\end{document}
